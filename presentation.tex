\documentclass[18pt]{beamer}

\usepackage{graphicx}
\usepackage{titlesec}
\usepackage{listings} % For inserting code
\usepackage[margin=0.5in]{geometry} % For adjusting the margin size
\usepackage{wrapfig}
\usepackage{mathtools}
\usepackage{multicol}
\usepackage{array}
\usepackage{calc} % For wrapping lines in a table
\usepackage{tkz-kiviat} % For inserting a Kiviat (radar chart) diagram
\usepackage{adjustbox} % For formatting of Kiviat diagram
\usepackage{color} % So that we can have some dank colours on the kiviat diagram
\usetikzlibrary{arrows}

% For the bibliography
\setbeamertemplate{bibliography entry title}{}
\setbeamertemplate{bibliography entry location}{}
\setbeamertemplate{bibliography entry note}{}

\newcommand{\LegendBox}[3][]{%
\xdef\fitbox{}%
\coordinate[#1] (LegendBox_anchor) at (#2) ;
    \foreach \col/\item [count=\hi from 0] in {#3} {
       \node[color = \col,draw,
             fill  = \col!50,
             minimum width  = 4 ex,
             minimum height = 2 ex,
             label={[anchor = left,name=b\hi]right:\item}] at ([yshift=\hi*4 ex]LegendBox_anchor) {};
             \xdef\fitbox{\fitbox(b\hi)}
   }%
 \node [draw,fit=\fitbox(LegendBox_anchor)] {};
}

\newcolumntype{L}{>{\let\newline\\\arraybackslash}m{#1}} % For tidier tables


\usetheme{metropolis}           % Use metropolis theme
\title{G54SOD}
\subtitle{Developing a Smart Campus}
\date{Spring 2017}
\author{Richard Davies, Elias Khoury, Rub\'{e}n Escocia, Abdullah Masud}
\institute{The University of Nottingham}
\begin{document}
    \graphicspath{ {images/} }
    \maketitle

    \begin{frame}{Introduction}
        \begin{columns}
            \column{0.5\textwidth}
                A smart campus is
                \begin{itemize}
                    \item Efficient
                    \item Safe
                    \item Sustainable
                    \item Responsive
                    \item Enjoyable Place to live and work
                    \item Underpinned and enhanced by digital or internet based technologies
                \end{itemize} \cite{Misc:uonsmartcampus}
            \column{0.5\textwidth}
                \begin{figure}
                \includegraphics[width=0.99\columnwidth]{smart}
                \caption{Can we make Nottingham University smart?}
                \end{figure}
        \end{columns}
    \end{frame}

    \begin{frame}{Feasibility Measures}
        \begin{adjustbox}{max totalsize={.9\textwidth}{.7\textheight},center}
        \begin{tikzpicture}
          \tkzKiviatDiagram{Coolness, Cost, Feasibility, Acceptability, Demand, Practicality, Adaptation, Integration, Expansion}
          \tkzKiviatLine[thick,
                       color      = orange,
                       mark       = none,
                       mark size  = 4pt,
                       opacity    = .2,
                       fill       = orange!20,
                       opacity=.5](8,7,8,7,7,7,6,8,7)
          \tkzKiviatLine[thick,
                        color      = red,
                        mark       = none,
                        ball color = red,
                        mark size  = 4pt,
                        opacity    = .5,
                        fill=red!20](5,4,8,5,6,7,8,8,2)
          \tkzKiviatLine[thick,
                        color      = brown,
                        mark       = none,
                        mark size  = 4pt,
                        opacity    = .5,
                        fill       = brown!20,
                        opacity=.5](6,7,7,6,6,6,7,6,5)
          \tkzKiviatLine[thick,
                       color      = blue,
                       mark       = none,
                       mark size  = 4pt,
                       opacity    = .5,
                       fill       = blue!20,
                       opacity=.5](5,2,6,4,4,6,1,1,1)
          \tkzKiviatLine[thick,
                       color      = green,
                       mark       = none,
                       mark size  = 4pt,
                       opacity    = .5,
                       fill       = green!20,
                       opacity=.5](5,7,5,5,3,6,6,5,7)
          \tkzKiviatLine[thick,
                       color      = purple,
                       mark       = none,
                       mark size  = 4pt,
                       opacity    = .5,
                       fill       = purple!20,
                       opacity=.5](3,3,4,4,3,2,2,2,2)

            \LegendBox[shift={(1cm,-1cm)}]{current bounding box.south east}%
            {red/Remodel Layout of Cafe,
             blue/Cafe Acoustics,
             green/Carpooling,
             purple/Preordering Student Lunches,
             pink/Parking Simulation,
             orange/Automated Waste Management}
         \end{tikzpicture}
     \end{adjustbox}
    \end{frame}

    \begin{frame}{Stakeholder Views}
        \begin{center}
        \begin{tabular}{|p{0.3\textwidth} | p{0.3\textwidth} | p{0.3\textwidth}|}
            \hline
        	Students & University Personnel & System Designers \\
            \hline
        	Main driving force of the waste system & Employ the waste staff & Discuss feasibility of the system \\
            \hline
            Desire a clean campus & Comprised of the waste staff & Responsible for maintenance \\
            \hline
            & Own the campus & \\
            \hline
        \end{tabular}
        \end{center}
    \end{frame}


    \begin{frame}{Replacing or Relocating}
        \begin{columns}
            \column{0.5\textwidth}
            \begin{itemize}
                \item The workload for the cleaning staff would be reduced, therefore fewer people would be required
                \item This situation might be beneficial for the university due to all the money that would be saved
                \item On the other hand, many families would stop receiving an income and relocating the cleaning staff might be a more ethical solution
            \end{itemize}
            \column{0.5\textwidth}
            \includegraphics[width=0.99\columnwidth]{humanvsrobot}
        \end{columns}
    \end{frame}


    \begin{frame}{Selection}
        \begin{columns}
            \column{0.5\textwidth}
                \begin{figure}
                \includegraphics[scale=0.25]{humanvsrobot}
                \end{figure}
            \column{0.5\textwidth}
                Each chromosome gets a $ \frac{1}{n} $ chance\\
                But better ones get a bigger probability
        \end{columns}
    \end{frame}

    \begin{frame}{Crossover}
        \begin{columns}
            \column{0.5\textwidth}
                \begin{figure}
                \includegraphics[scale=0.3]{family}
                \end{figure}
            \column{0.3\textwidth}
                One point\\
                Two point\\
                K point\\
                Uniform
        \end{columns}
    \end{frame}

    \begin{frame}{Culling}
        \begin{columns}
            \column{0.5\textwidth}
                Keep the best
            \column{0.5\textwidth}
                \begin{figure}
                \includegraphics[scale=1]{chicken}
                \end{figure}
        \end{columns}
    \end{frame}


    \begin{centering}
        \begin{frame}[c]{}
            \frametitle{Section 8}
            Problems
        \end{frame}
    \end{centering}

    \begin{frame}{Problems}
        \begin{enumerate}
            \item Chromosomal representation
            \item Local Maxima
            \item Heuristics are good
        \end{enumerate}
    \end{frame}

    \begin{centering}
        \begin{frame}[c]{}
            \frametitle{Section 8}
            But, why?
        \end{frame}
    \end{centering}

    \begin{frame}[allowframebreaks]
            \frametitle{References}
            \bibliographystyle{plain}
            \bibliography{presentation}
    \end{frame}

\end{document}
